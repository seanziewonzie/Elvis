\documentclass[12pt]{article}

\input ElvisMacros.mac

\begin{document}

%\pagenumbering{gobble}

\begin{center}
{\LARGE\bf\underline{Exercises on the Elvis Problem - Part I}}
\end{center}

\section{Preliminaries}

{\blue
\begin{exer}
Suppose $f:\R^n\to\overline{\R}$.
\begin{itemize}
	\item[(a)]  The {\em effective domain} of $f(\cdot)$ is defined by $\dom(f):=\{x:f(x)<\infty\}$.  Show if $f(\cdot)\in\cF$, then $		\dom(f)$ is a convex set. 
	\item[(b)]  If $\dom(f)=\R^n$, show that $f\in\cF$ if and only if 
		\begin{equation}\label{eq: cnvx}
			f(x_{\lambda})\leq \lambda f(x_0)+(1-\lambda)f(x_1)
		\end{equation}
		for all $x_0,x_1\in\R^n$ and $0\leq \lambda\leq 1$, and where $x_{\lambda}:=\,\lambda\ x_0+\,    (1-\lambda)x_1$.
	\item[(c)]  Define an arithmetic and an order relation on $\overline{\R}$ so that the property that a lsc $f(\cdot)$ is convex is 		characterized by (1).
\end{itemize}
\end{exer}
}


\begin{itemize}
%
\item[(a)] 
\begin{proof}
Assume that $f\in\cF$. 
Let $0\leq\lambda\leq 1$ and $x_0, x_1 \in \dom(f)$. 

Then since $\epi(f)$ is convex,
\begin{eqnarray*}
(x_0,f(x_0)),(x_1,f(x_1))&\in&\epi(f) \\
\lambda(x_0,f(x_0))+(1-\lambda)(x_1,f(x_1)&\in&\epi(f)\\
(\lambda x_0+ (1-\lambda)x_1, \lambda f(x_0)+(1-\lambda)f(x_1))&\in&\epi(f) 
\end{eqnarray*}
%
By definition of $\epi(f)$, $$f(\lambda x_0+(1-\lambda)x_1) \leq \lambda f(x_0)+(1-\lambda)f(x_1).$$
Since $x_0$, $x_1 \in \dom(f)$, $f(x_0)<\infty$ and $f(x_1)<\infty$ so that
$$f(\lambda x_0+(1-\lambda)x_1)<\infty.$$
Therefore, $\lambda x_0 +(1-\lambda)x_1\in\dom(f)$ so $\dom(f)$ is a convex set. 
\end{proof}
%
\pagebreak
\item[(b)] 
\begin{proof}\quad\\ Let $\dom(f)=\R^n$.\\
($\Rightarrow$) Assume that $f(\cdot)\in\cF$. Let $0\leq\lambda\leq 1$ and $x_0, x_1\in \mathbb{R}^n$.
Since $\dom(f)=\R^n$, $f(x_0), f(x_1) <\infty$. Then
\begin{eqnarray*}
(x_0,f(x_0)),(x_1,f(x_1))&\in&\epi(f) \\
(\lambda x_0+ (1-\lambda)x_1, \lambda f(x_0)+(1-\lambda)f(x_1))&\in&\epi(f)
\end{eqnarray*}
By definition of the $\epi(f)$,
$$f(x_\lambda) = f(\lambda x_0+(1-\lambda)x_1)\leq\lambda f(x_0)+(1-\lambda)f(x_1).$$ 
\\
($\Leftarrow$) 
Assume that $f(x_\lambda) \leq(1-\lambda)f(x_0)+\lambda f(x_1).$
Let $0\leq\lambda\leq 1$ and $x_0,x_1\in\mathbb{R}^n$.
Since $\dom(f)=\R^n$, $f(x_0), f(x_1) <\infty$. 
Then $(x_0,f(x_0)),(x_1,f(x_1))\in\epi(f)$ so that $\epi(f)\neq\emptyset$ meaning $f(\cdot)$ is proper. 
Using the assumption, 
$$f(x_\lambda)\leq\lambda f(x_0)+ (1-\lambda)f(x_1)$$
Then $\epi(f)$ is convex since 
$$(x_\lambda ,\lambda f(x_0)+(1-\lambda)f(x_1))\in\epi(f).$$
To see that $f(\cdot)$ is lsc consider the sequence $(x_k,f(x_k))\in\epi(f)$ such that $$(x_k,f(x_k))\rightarrow(\overline{x},f(\overline{x})).$$
Since $\overline{x}\in\mathbb{R}^n$ we have that $\overline{x}\in\dom(f)$ so that $f(\overline{x})<\infty$.
Therefore, $(\overline{x},f(\overline{x}))\in\epi(f)$ so $\epi(f)$ is closed and $f(\cdot)$ is lsc.
It follows that $f(\cdot)\in\cF$.


\end{proof}
%
%
\item[(c)]  Rules of arithmetic on $\overline{\R}$ (to be added to the usual arithmetic rules)
\begin{align*} 
	&\alpha + \infty = \infty + \alpha = \infty \text{ for } -\infty<\alpha\leq\infty, \\
	&\alpha - \infty = -\infty + \alpha = -\infty \text{ for } -\infty<\alpha\leq\infty, \\
	&\alpha\infty=\infty\alpha=\infty, \alpha(-\infty)=(-\infty)\alpha=-\infty \text{ for } 0\leq\alpha\leq\infty \\
	&\alpha\infty=\infty\alpha=-\infty, \alpha(-\infty)=(-\infty)\alpha=\infty \text{ for } -\infty<\alpha<0\\
	&\infty-\infty = -\infty+\infty = \infty
\end{align*}
The order relation on $\overline{\R}$ is $-\infty\leq\alpha\leq\infty$ for every $\alpha\in\overline{\R}$.
\end{itemize}

\pagebreak

{\blue
\begin{exer}  Suppose $S\subseteq \R^n$.  Show $I_S(\cdot)$
\begin{itemize}
	\item[(a)] is lsc if and only if $S$ is closed;
	\item[(b)] is a convex function if and only if $S$ is a convex set; and
	\item[(c)] belongs to $\cF$ if and only if $S$ belongs to $\cC$.
\end{itemize}
\end{exer}
}

\begin{itemize}
\item[(a)] 
\begin{proof}\quad\\
($\Rightarrow$) 
Let $I_S(\cdot)$ be lsc, then $\epi(I_S)$ is closed. 
Consider the sequence $(x_k,r_k)\in\epi(I_S)$ with $(x_k,r_k)\rightarrow(\overline{x},\overline{r})$.
Then $(\overline{x},\overline{r})\in \epi(I_S)$ so that $I_S(\overline{x})\leq \overline{r}\in \mathbb{R}$.
Then $I_S(\overline{x})=0$ so that $\overline{x}\in S$. By similar reasoning, the sequence $x_k\in S$ and since $x_k\rightarrow\overline{x}$, $S$ is closed.
\\\\
($\Leftarrow$) 
Let S be closed. Consider the sequences $x_k\in S$ with $x_k\rightarrow \overline{x}$ and $r_k\geq 0$ with $r_k\rightarrow \overline{r}\geq 0$. 
Then $\overline{x}\in S$ and $(x_k,r_k),(\overline{x},\overline{r})\in\epi(I_S)$ such that $(x_k,r_k)\rightarrow (\overline{x},\overline{r})$. Therefore $\epi(I_S)$ is closed so that $I_S$ is lsc.
\end{proof}
\item[(b)] 
\begin{proof}\quad\\
Recall that $x_{\lambda}:=(1-\lambda)\,x_0+\lambda\,x_1$.\\
($\Rightarrow$) Let $I_S(\cdot)$ be convex function, $0\leq\lambda\leq 1$, and $x_0,x_1\in S$. 
Since $I_S(\cdot)$ is convex, $\epi(I_S)$ is convex and
$(x_0,r_0),(x_1,r_1)\in \epi(I_S)$ for $r_0,r_1\geq 0$.
So that $(x_\lambda,\lambda r_0+(1-\lambda)r_1)\in \epi(I_S).$ 
Then 
\[
I_S(x_\lambda) \leq \lambda r_0+(1-\lambda)r_1 \Rightarrow
I_S(x_\lambda) = 0 \Rightarrow
x_\lambda \in S.
\]
So $S$ is a convex set. \\\\
($\Leftarrow$) Let $0\leq\lambda\leq 1$. 
Assume $S$ is a convex set with $x_0,x_1\in S$ and $(x_0,r_0),(x_1,r_1)\in\epi(I_S)$.
Since $S$ is convex, $$x_\lambda\in S \Rightarrow I_S(x_\lambda)=0.$$
Furthermore, $I_S(x_0)=0$ and $I_S(x_1)=0$. Then $0 \leq r_0,r_1$ so that $0\leq \lambda r_0 + (1-\lambda)r_1$. 
Then we have that $$I_S(x_\lambda) \leq \lambda r_0 + (1-\lambda)r_1$$ so that
$(x_\lambda, \lambda r_0 + (1-\lambda)r_1)\in \epi(I_S)$ meaning $\epi(I_S)$ is convex so that $I_S(\cdot)$ is a convex function.
\end{proof}

\item[(c)] \begin{proof} \quad \\ Assume that $S\subseteq \mathbb{R}^n$. \\
($\Rightarrow$)  Let $I_S(\cdot)\in\cF$ so that $I_S(\cdot)$ is lsc, convex and proper. In order to show that $S\in\cC$, must show that $S$ is closed, convex, and nonempty. By parts (a) and (b), $S$ is closed and convex. Since $I_S(\cdot)$ is proper, $\epi(I_S)\neq\emptyset$ or there exists $(x,r)\in\epi(I_S)$ such that $r\geq I_S(x)$. Then $I_S(x)\neq\infty$ so $x\in S$. 

($\Leftarrow$) Let $S\in\cC$ so that $S$ is closed, convex, and nonempty. By parts (a) and (b), $I_S(\cdot)$ is lsc and convex. 
To show that $I_S(\cdot)$ is proper, let $x\in S$. Then for every $r\geq0$, $I_S(x)=0\geq r$ meaning $(x,r)\in\epi(I_S)$.
\end{proof}
\end{itemize}

{\blue
\begin{exer}
Show that $F$ is closed and convex if and only if
\[
F=\bigcap\biggl\{\cH_{\vec{n},r}:\;\vec{n}\in\R^n,\,r\in\R\text{  are such that } F\subseteq \cH_{\vec{n},r} \biggr\}
\]
\end{exer}}

\begin{proof} \quad\\
($\Rightarrow$) 
Assume that $F$ is closed and convex. Define the following 
\[
H := \bigcap\biggl\{\cH_{\vec{n},r}:\;\vec{n}\in\R^n,\,r\in\R\text{  are such that } F\subseteq \cH_{\vec{n},r} \biggr\}
\]
Let $v$ be arbitrary such that $v\not\in F$. 
Since $F$ is closed, we can find $x=\proj_F(v)$.
Then by the Separation Theorem, for each $v$ there exists $\vec n\in\R^n$ such that 
\begin{equation*}
	\sup\bigl\{\langle v',\vec n\rangle:\; v'\in F\bigr\} < \langle v,\vec n\rangle.
\end{equation*}
In particular, $\vec n = \frac{v-x}{2}$.
Then the half space that separates $v$ from $F$ is 
\[
\cH_{\vec{n},r}:=\bigl\{w :\;\langle\vec{n},w\rangle\leq r\bigr\}.
\]
Where $r=\langle\vec{n},\frac{v+x}{2}\rangle$ so that $F\subseteq\cH_{\vec{n},r}$.
%(You can verify this defines the desired half-space in $\mathbb{R}^2$ by considering the geometric formulation of the inner product, i.e. 
$\langle\vec{n},w\rangle = \Vert\vec{n}\Vert\Vert w\Vert\cos\theta$.)
Since $F\subseteq\cH_{\vec{n},r}$ for every such $\cH_{\vec{n},r}$,
$$F\subseteq H.$$
Furthermore, since above we showed that for every $v\not\in F$ we can find a $\cH_{\vec{n},r}\not\ni v$ we have 
$F^c\cap H=\emptyset.$
Thus $F=H$.\\\\
($\Leftarrow$) 
Assume that $F = H$. 
Let $x_0,x_1\in \cH_{\vec{n},r}$ for some $\vec{n}\in\R^n,\,r\in\R\text{  such that } F\subseteq \cH_{\vec{n},r}$. 
Then $\langle\vec{n}, x_0\rangle\leq r$ and $\langle\vec{n},x_1\rangle\leq r$ so that
\begin{eqnarray*}
\langle\vec{n},\lambda x_0+(1-\lambda)x_1\rangle &=& \lambda\langle\vec{n}, x_0\rangle+(1-\lambda)\langle\vec{n},x_1\rangle \\
	&\leq& \lambda r+(1-\lambda)r \\
	&=& r
\end{eqnarray*}
Therefore $\lambda x_0+(1-\lambda)x_1\in  \cH_{\vec{n},r}$ so that each $ \cH_{\vec{n},r}$ is convex. 
Since $F$ is the intersection of convex sets, it itself is convex.
Similarly since each half space is closed, the intersection of all half spaces is closed.  Thus $F$ is both convex and closed.

\end{proof}


\end{document}









